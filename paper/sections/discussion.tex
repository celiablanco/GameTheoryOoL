\section*{Discussion}

The results identify a sharp threshold separating qualitatively distinct regimes of chemical organisation, consistent with the broader role of thresholds as organising principles in origin-of-life scenarios. In many prebiotic contexts, small and continuous changes in parameters can produce discontinuous changes in system-level behaviour, marking transitions between regimes with fundamentally different modes of persistence and control \cite{Jeancolas2020Threshold}. In the present model, the relevant threshold concerns the persistence of population-environment feedback. Below a critical strength, environmental modification remains transient and does not alter long-term outcomes. Above it, modifier-producing populations can collectively stabilise the conditions that support their own persistence.

From a chemical perspective, this threshold reflects a balance between production and dissipation of environmentally active species. The environmental variable represents the concentration of a modifier produced by rule-makers and removed by background processes such as dilution, degradation, or buffering. When production exceeds dissipation, the modifier accumulates and reshapes the effective growth conditions experienced by the population. When dissipation dominates, environmental improvements fail to persist. The transition therefore marks a shift from environments that merely respond to chemistry to environments that are actively stabilised by it.

Importantly, this transition does not concern the fidelity of replication, the closure of reaction networks, or the stability of specific molecular structures. Instead, it concerns the stability of environmental influence itself. In threshold-based frameworks for the origin of life, this corresponds to a system-environment coupled transition, in which persistence depends jointly on internal population dynamics and environmental timescales \cite{Jeancolas2020Threshold}. The emergence of rule-making chemistry thus represents a qualitative change in how chemical systems relate to their surroundings, from passive dependence to partial environmental control. This interpretation becomes explicit in the chemically grounded formulation of the model. There, the environmental variable corresponds to the concentration of a concrete modifier species produced by rule-makers and removed by background processes. The effective feedback strength reflects the balance between how strongly the modifier enhances replication and how long it persists in the environment. When production outpaces loss, the modifier accumulates to levels that feed back to enhance the replication of its producers. When loss dominates, environmental effects remain short-lived. The minimal reaction network therefore provides a concrete chemical mechanism by which replicators can create, maintain, and transmit enabling environmental conditions without requiring genetic encoding or complex metabolism.

Recent work has shown that the emergence of Darwin-like evolutionary dynamics may itself require crossing a threshold separating degradation-dominated chemistry from sustained propagation \cite{Kocher2024PrebioticEmergence}. In that work, cooperative autocatalysts are shown to overcome such a barrier and enter a growth-dominated regime in which evolutionary invasion becomes possible. Related perspectives argue more broadly that ecological and evolutionary dynamics form a continuum even in purely chemical systems, with selection acting on persistence and dispersal prior to genetic encoding \cite{Baum2023Continuum}.The transition identified here is conceptually distinct. Replication is assumed to be already possible, but the results show that replicating populations may nevertheless remain confined to a rule-taking regime unless population-environment feedback persists strongly enough to stabilise favourable conditions. In this sense, rule-making chemistry represents a subsequent threshold, marking the shift from propagation under given conditions to partial control over the conditions that govern propagation itself.

The model also connects naturally to niche-construction and constraint-inheritance frameworks \cite{Laland1999Niche}. Modifier-producing replicators generate environmental conditions that persist long enough to influence the selective environment of subsequent generations, providing a minimal example of non-genetic inheritance. Once established, these conditions stabilise the modifier-dominated regime, much as niche construction stabilises organism-environment couplings in biological systems. In this sense, the model illustrates how chemical populations can acquire a rudimentary form of autonomy by reinforcing the constraints that favour them.

A central consequence of the threshold is strong history dependence. Initial conditions jointly determine whether the system crosses the separatrix into the modifier-dominated state or relaxes to the rule-taking state. Such sensitivity to initial conditions may have been important on the early Earth, where heterogeneous microenvironments would have intermittently supported the accumulation of environmental modifiers. Only locales that crossed the threshold would have supported self-maintaining chemical organisation, potentially seeding the emergence of more complex evolutionary dynamics. At the same time, the model clarifies why such transitions should be rare rather than inevitable. Environmental modification is widespread in chemistry, but most modifications are too weak, too transient, or too costly to persist long enough to reshape selection. In the present framework, modifiers that decay rapidly are erased before they can influence population composition, while modifiers that enhance replication only weakly fail to offset their production costs. Even when bistability exists, successful transition into the modifier-dominated state requires crossing a separatrix that depends jointly on population composition and environmental state. The model therefore does not imply that feedback generically produces autonomy. Instead, it shows that autonomy emerges only under restricted conditions of sufficient feedback strength, persistence, and favourable history.

Despite its simplicity, the model suggests qualitative experimental predictions. In laboratory reactors or serial-dilution experiments, one could compare a resource-consuming replicator with a variant that also produces a stabilising modifier. The theory predicts bistability when environmental modification is strong relative to cost, with hysteresis under controlled perturbations providing evidence for the predicted feedback threshold. More generally, the framework highlights population-environment feedback as a distinct axis along which chemical systems can transition from passive adaptation to partial environmental control.

In summary, the model isolates a key ingredient in the emergence of autonomy. Early evolutionary dynamics may have depended not only on replication with variation and selection, but also on the capacity of chemical systems to stabilise the constraints enabling their own persistence. The results suggest that the transition from rule-taking to rule-making chemistry is neither inevitable nor ubiquitous, but rare and conditional, arising only when feedback is strong enough to persist across the timescales that shape population change.