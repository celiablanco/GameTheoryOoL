\section{Introduction}

The origin of life is often described as the emergence of chemical systems capable of sustaining organisation, reproduction, and adaptive change. Classical models illuminate different ways in which such organisation might arise and persist. For example, autocatalytic sets show how reaction networks can collectively maintain and propagate themselves \cite{Kauffman1986Autocatalytic}; hypercycle models demonstrate how cooperating replicators can overcome parasitic alternatives once template-based heredity is present \cite{Eigen1977}; and protocell constructs illustrate how compartmentalisation can couple metabolism, growth, and division into integrated units \cite{Szostak2001}. Despite capturing essential features of early evolution, these frameworks typically focus on the dynamics of replicating entities themselves and treat reaction conditions as externally imposed and fixed, rather than shaped or stabilised by the chemical systems in which replication occurs.

However, life is distinguished by its capacity to modify the conditions under which it operates \cite{OdlingSmee2003}. Niche construction \cite{Laland1999Niche} and ecological scaffolding \cite{Black2020} show that living systems shape aspects of their environment in ways that influence subsequent evolutionary dynamics. Origins of life research likewise suggests that early chemistries may have begun to stabilise favourable local conditions rather than merely endure externally imposed ones. In particular, recent work on autocatalytic chemical ecosystems emphasises the continuity between ecological and evolutionary dynamics in open chemical systems, showing how feedback between reaction networks and their environments can support persistence and adaptive change prior to genetic inheritance \cite{Baum2023Continuum,Kalambokidis2024EcoEvo}.

This convergence suggests that two broad classes of constraints are central to the emergence of life. The first concerns reliable replication and information preservation, including fidelity, stability, and protection of replicating structures. The second concerns the ability of chemical systems to modify or stabilise the conditions under which replication occurs. Much of the origin-of-life literature has focused on the former, developing detailed accounts of how inheritance and replication might become robust. By contrast, the latter has received comparatively less attention, particularly in pre-genetic contexts where regulation and environmental coupling must precede encoded heredity. This emphasis aligns with discontinuist perspectives, which argue that the transition from molecular evolution to biological organisation is marked not by replication alone, but by the emergence of regulatory mechanisms that integrate system and environment \cite{BichDamiano2012Emergence}. Understanding when and how such endogenous feedback can first arise in chemical populations is therefore essential for explaining the transition from rule-taking to rule-making chemistry.

Here we develop a minimal population-environment feedback model to examine when environment-modifying activity, once present, can become self-sustaining and reshape the conditions governing chemical propagation. We analyse the conditions under which such rule-making strategies can spread, how their fate depends on initial abundance and environmental state, and how even a single dynamical environmental variable reshapes the phase structure governing long-term outcomes.
